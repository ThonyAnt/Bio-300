\documentclass[12pt]{article}
\usepackage{array,booktabs}
\usepackage{graphicx} % Required for inserting images
\usepackage{setspace}
\usepackage[margin=2.5cm]{geometry} % margins
\usepackage[parfill]{parskip}
\usepackage{enumitem}
\usepackage{amsfonts,latexsym,amsthm,amssymb,amsmath,amscd,euscript}

%Chemistry Packages
\usepackage{chemfig}
\usepackage{mhchem}
\usepackage{chemformula}
\usepackage{siunitx}
\sisetup{group-digits=false}
\usepackage{cancel}

\setlength{\parindent}{0pt}
\AtBeginDocument{\setstretch{1.125}}

\title{Hemodialysis Paragraph}
\author{Anthony Yu}

\begin{document}

%Some new commands
\newcommand{\problem}[1]{\subsection*{Problem {#1}}}
\newenvironment{enumAlph}{\begin{enumerate}[label=(\alph*)]}{\end{enumerate}}

\makeatletter
\newcommand{\skipitems}[1]{%
\addtocounter{\@enumctr}{#1}%
}
\makeatother

\newcommand{\chunit}[3]{\qty{#1}{{#2}\,\ce{#3}}}
\newcommand{\chuniteval}[3]{\qty[evaluate-expression]{#1}{{#2}\,\ce{#3}}}

\newtheorem{definition}{Definition}

\maketitle

Hemodialysis is a treatment method for patients with kidney failure by removing extra
fluids and waste from the blood stream. It helps regulate mineral levels (potassium,
sodium, calcium) as well as blood pressure. Hemodialysis
works by connecting the patient's blood vessels to a dialyzer filter through
an incision. A Heparin pump pumps the blood into the dialyzer filter to 
prevent clotting. Inside the filter, there is a side for blood and a side
for dialyzer fluid, separated by a membrane. Harmful materials in the blood
diffuse through the membrane while cells that originally belong to the body 
remain in the blood. The newly filtered blood is then transferred back into 
the patient's body. 

\end{document}