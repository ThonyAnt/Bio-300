\documentclass[12pt]{article}
\usepackage{array,booktabs}
\usepackage{graphicx} % Required for inserting images
\usepackage{setspace}
\usepackage[margin=2.5cm]{geometry} % margins
\usepackage[parfill]{parskip}
\usepackage{enumitem}
\usepackage{amsfonts,latexsym,amsthm,amssymb,amsmath,amscd,euscript}

%Chemistry Packages
\usepackage{chemfig}
\usepackage{mhchem}
\usepackage{chemformula}
\usepackage{siunitx}
\sisetup{group-digits=false}
\usepackage{cancel}

\setlength{\parindent}{0pt}
\AtBeginDocument{\setstretch{1.125}}

\title{Test 3 Review}
\author{Anthony Yu}

\begin{document}

%Some new commands
\newcommand{\problem}[1]{\subsection*{Problem {#1}}}
\newenvironment{enumAlph}{\begin{enumerate}[label=(\alph*)]}{\end{enumerate}}

\makeatletter
\newcommand{\skipitems}[1]{%
\addtocounter{\@enumctr}{#1}%
}
\makeatother

\newcommand{\chunit}[3]{\qty{#1}{{#2}\,\ce{#3}}}
\newcommand{\chuniteval}[3]{\qty[evaluate-expression]{#1}{{#2}\,\ce{#3}}}

\newtheorem{definition}{Definition}

\maketitle

\begin{definition}[First Law of Thermodynamics]
    Energy cannot be creaed nor destroyed. 
\end{definition}
\begin{definition}[Second Law of Thermodynamics]
    When energy gets converted from one to another, the amount of useful energy decreases. 
\end{definition}
An example would be that gasoline cars have only $20\%$ efficiency. 
\begin{definition}[Entropy]
    Tendency towards the loss of orderliness and the loss of useful energy.
    An example would be something at equilibrium. 
\end{definition}

\section{ATP}
\begin{definition}[ATP]
    Nitrogen containing base adenine + sugar ribose base + 3 phospohate groups
\end{definition}
Why is breaking ATP exergonic overall? Shouldn't breaking bonds consume energy?
Remember that we're breaking it through hydrolysis. It's true that
breaking bonds take some energy, but the products that form when combined
recombined with water (ADP and Pi) are much more stable, thus releasing a ton of energy. 

Kinase P transfer enzyme. Just remember it. 

\subsection{3 ways ATP does work in cell}
\begin{enumerate}
    \item Coupled reaction: use the glutamic acid example (see beneath)
    \item Active transport (such as protein pumps). 
    \item Movement: Motor protein, flagellum uses ATP. Phosphate group changes shit. 
\end{enumerate}

\section{Coupled reaction}
An example would be glutamic acid conversion to Glutamine. It is in and
of itself a endogonic reaction. However, with ATP, glutamine acid
forms an intermediate with ADP + Pi,  

\section{Enzymes}
\subsection{How does it work}
\begin{enumerate}
    \item Induced fit model. 
    The active site is a pocket on the protein that has projecting 
    R groups (from the amino acid backbone) that forms H bonds, 
    ionic bonds, or temporary covalent bonds with the substrate. 
    The enzyme slightly changes shape as the substrate
    enters the active site, putting strain of its bonds, therefore
    reducing the activation energy.
    \item Orientation: The enzyme helps position the substrates in 
    their right 3D orientations, increasing the number of 
    productive collisions. 
\end{enumerate}

\subsection{Specificity}
The substrate must fit the active site in terms of size, shape, and 
charge compatibility. 

\subsection{5 ways cells regulate enzyme activity}
\begin{enumerate}
    \item Turn on/off genes that code enzymes.
    \item Synthesized in inactive forms.
    \item Competitive inhibition
    \item Non-competitive inhibition (also known as allosteric regulation I think?)
    \item Degradation
\end{enumerate}

\subsection{Environment}
pH affects enzymes ince the h-bonds crucial to their 3D structure
are stable in only narrow ranges of pH. 

Moderatly high temperatures optimize rxns. If the temp is too low, 
molecular motion decreases. If temp too high, h-bonds might 
be broken due to excessive motion, causing denaturation. 

\section{Peroxisomes}
They produce \ch{H2O2} as part of its natural metabolic reactions.
However it also breaks them down instantly, to prevent
harm to the cells. 

\section{Glycolysis}
Glucose turns into 2 pyruvate, gaining 2 net ATP, 2 net \ch{NADH} in the process.

\section{Intermediate Step}
2 pyruvate reacts with coenzyme A to 
form 2 acetyl coenzyme A.
(It gets used in krebs cycle, and converts back
into coenzyme A). Each pyruvate produces a \ch{CO2}
and \ch{NADH}.

\section{Krebs Cycle}
Takes the acetyl coenzyme A and churns out a bunch of shit. The
acetyl coA turns back into coA at the end. 
Releases 2 \ch{CO2}
and 3\ch{NADH} and 1 \ch{FADH2} and 1 ATP for each acetyl coA.

\section{Fermentation}

\subsection{CH Bonds}
Chisholm calls them "high energy bonds" not because they're hard to break.
They are actually unstable; thus when broken they have the potential
to form more stable bonds and release a lot of energy. 

Without \ch{O2}, \ch{NAD+} cannot naturally regenerate 
since the \ch{H+} cannot be disposed of. 
Fermentation can help regenerate \ch{NAD+} allowing us to 
re-enter glycolysis. 

\section{ETC}

\subsection{Coenzyme Q and cytochrome C}
They are membrane bound and transports electrons between protein
complexes. Diffusion is faster since they're membrane bound. 

\subsection{Proteins}
Protein complex 1,3,4 pump protons, but protein complex 2 only
picks up electrons.

\end{document}