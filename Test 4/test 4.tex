\documentclass[12pt]{article}
\usepackage{array,booktabs}
\usepackage{graphicx} % Required for inserting images
\usepackage{setspace}
\usepackage[margin=2.5cm]{geometry} % margins
\usepackage[parfill]{parskip}
\usepackage{enumitem}
\usepackage{amsfonts,latexsym,amsthm,amssymb,amsmath,amscd,euscript}

%Chemistry Packages
\usepackage{chemfig}
\usepackage{mhchem}
\usepackage{chemformula}
\usepackage{siunitx}
\sisetup{group-digits=false}
\usepackage{cancel}

\setlength{\parindent}{0pt}
\AtBeginDocument{\setstretch{1.125}}

\title{Test 4 Review}
\author{Anthony Yu}

\begin{document}

%Some new commands
\newcommand{\problem}[1]{\subsection*{Problem {#1}}}
\newenvironment{enumAlph}{\begin{enumerate}[label=(\alph*)]}{\end{enumerate}}

\makeatletter
\newcommand{\skipitems}[1]{%
\addtocounter{\@enumctr}{#1}%
}
\makeatother

\newcommand{\chunit}[3]{\qty{#1}{{#2}\,\ce{#3}}}
\newcommand{\chuniteval}[3]{\qty[evaluate-expression]{#1}{{#2}\,\ce{#3}}}

\newtheorem{definition}{Definition}

\maketitle

\section{Cancer}
\subsection{9 Hallmarks}
Making more cells isn't enough for cancer to proliferate. 
These cells start by overdividing; but they will quickly
get killed by the various regulatory checkpoints in our body,
such as immune system or apoptosis. Thus they need to satisfy
all the hallmarks. 

\begin{enumerate}
    \item Proto-oncogens mutate and become hyperactive: 
    proto-oncogenes are genes that can turn into a cancer gene (AKA
    an oncogene). They usually help the cell go through cell cycle. 
    An example is rtk, a protein associated with a proto-oncegene. (recall 
    that it is the stick thing that has 3 branches, which combine 
    to form a dimer when they receive a growth signal). Once the 
    proto-oncogene becomes mutated, rtk will keep telling the cell to divide.

    See the sheet for more details on this process. 
    \item Anti-oncagenes mutate and become inhibited: 
    For example, prb and p53 genes, which are master breaks of the cell cycle,
    stop working.
    
    \item Apoptotic genes mutate. This blocks apoptosis and fail to trigger it.
    Example would be bcl-21 proteins block apoptosis, when they activate
    the cells no longer suicide. 
    \item Telomerase genes become active and extend telomeres to help cells
    bypass hayflick limit and make them immortal
    \item Cells must acquire the ability to produce VEGF and other growth 
    factors that stimulate growth of blood vessels to nourish tumor
    (a process sustained angiogenesis)
    \item Metastatis: cancer cells separate from their intercellular junctions
    and invade other tissues. 
    \item Changes to metabolism. Some cells do 
    anaerobic glyclyosis even in presence of oxygen. 
    \item Avoiding immune destruction. Straightforward.
    \item Phenotypic plasticity: 
    Cancer cells de-differentiate and return to a more stem-cell
    like state. 
\end{enumerate}

\subsection{Causes of mutation}
\begin{enumerate}
    \item Random mutations due to replication errors
    \item Ionizing radiation
    \item Mutagenic materials, such as arsenic in water
    \item Viral infections promote proto-oncogenjes or 
    deactivate tumor suppressor genes. HPV. 
\end{enumerate}

\subsection{Cancer in old age}
most peopole have cancer in old age because the longer you live,
the more likely there is for a mutation to occur. It's a chance game.

\subsection{Early cancer}
\begin{enumerate}
    \item inheritance of mutated proto-oncogenes (thus present in all cells!)
    \item Inheritance of dfective DNA repair genes
    \item different versions of detoxifying enzymes that can 
    make more/less harmful molecules from original molecules, or work faster/slower
    \item Impaired immune system
    \item Exposure to carcinogens 
    \item disease
\end{enumerate}

\begin{definition}[Carcinogens]
    Covalent bonds between carcigones and DNA cause mutations
\end{definition}

\end{document}