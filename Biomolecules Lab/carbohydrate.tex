\documentclass[12pt]{article}
\usepackage{array,booktabs}
\usepackage{graphicx} % Required for inserting images
\usepackage{setspace}
\usepackage[margin=2.5cm]{geometry} % margins
\usepackage[parfill]{parskip}
\usepackage{enumitem}
\usepackage{amsfonts,latexsym,amsthm,amssymb,amsmath,amscd,euscript}

%Chemistry Packages
\usepackage{chemfig}
\usepackage{mhchem}
\usepackage{chemformula}
\usepackage{siunitx}
\sisetup{group-digits=false}
\usepackage{cancel}

\setlength{\parindent}{0pt}
\AtBeginDocument{\setstretch{1.125}}

\title{Carbohydrate Lab}
\author{Anthony Yu}
\date{September 2024}

\begin{document}

%Some new commands
\newcommand{\problem}[1]{\subsection*{Problem {#1}}}
\newenvironment{enumAlph}{\begin{enumerate}[label=(\alph*)]}{\end{enumerate}}

\makeatletter
\newcommand{\skipitems}[1]{%
\addtocounter{\@enumctr}{#1}%
}
\makeatother

\newcommand{\chunit}[3]{\qty{#1}{{#2}\,\ce{#3}}}
\newcommand{\chuniteval}[3]{\qty[evaluate-expression]{#1}{{#2}\,\ce{#3}}}

\maketitle

\section*{Station 1}
The molecular formulas for glucose, frucrtose and sucrose are $\ch{C6H12O6}$, $\ch{C6H12O6}$, and $\ch{C12H22O11}$ respectively. 
The molecular formula for glucose and fructose are the same, 
but the arrangement of the atoms are different. One such difference is that
glucose has a 6-carbon ring, while fructose has a 5-carbon ring. Additionally, glucose has
aldehyde group while fructose has ketone group. 

\section*{Station 2}
\subsection*{Lab work}
See lab sheet.

\subsection*{Write up}
The reason why these sugars have different tastes, and we might prefer some over others, is because of the way humans have evolved to adapt to 
different types of sugars. Usually, sweeter sugars are more nourishing, as the brain
is wired to reward us to eat more of them. This is why fructose, found in fruits,
is sweet, but lactose isn't, as it is found in milk, something that we don't usually eat 
after infancy. Maltose is mainly found in barley, something that we don't usually eat,
which makes it less sweet.

In order to make a disaccharide from glucose and galactose, we need to remove a water molecule through dehydration synthesis. 
This is necessary for the glucose and galactose to bond together, as it allows the 
two molecules to form a stable bond through the remaining oxygen. 

\section*{Station 3}
\subsection*{Lab work}
The sugar bubbles up and begins melting. The liquid gets darker 
as it continues to heat up, and eventually turns into a black solid, which ignites 
when the flame is brought close to it.

\subsection*{Write up}
The black residue left is burnt sugar containing carbon. 
Since sucrose is $\ch{C6H12O6}$ (containing hydrogen, oxygen, and carbon), 
it will release $\ch{CO2}$ and $\ch{H2O}$ when it burns.

The heat and light come from the formation of water molecules + carbon dioxide, which releases energy. This
is because these bonds are more stable than the bonds in the sugar, since the electrons
are next to a more electronegative atom (oxygen), making them closer to the nucleus.

\section*{Station 4}
\subsection*{Lab work}
See lab sheet.

\subsection*{Write up}
The KI reacted with starch, cellulose, and glycogen, but not with glucose.

The structures of starch, cellulose, and glycogen are all polysaccharides, 
which are long chains of glucose subunits. However, cellulose is made of 
$\beta$-glucose, and starch and glycogen are made of $\alpha$-glucose.

Th water tests make for the control group. They are used to show that
the KI, instead of the water, is reacting with the polysaccharides.

The amalyse changed starch and glycogen by breaking them down into glucose. 
I know this because the diastix's color changed from blue to green, indicating
a presence of glucose. 

The amalyse facilitated the process of hydrolysis. 

\end{document}