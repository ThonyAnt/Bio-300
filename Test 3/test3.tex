\documentclass[12pt]{article}
\usepackage{array,booktabs}
\usepackage{graphicx} % Required for inserting images
\usepackage{setspace}
\usepackage[margin=2.5cm]{geometry} % margins
\usepackage[parfill]{parskip}
\usepackage{enumitem}
\usepackage{amsfonts,latexsym,amsthm,amssymb,amsmath,amscd,euscript}

%Chemistry Packages
\usepackage{chemfig}
\usepackage{mhchem}
\usepackage{chemformula}
\usepackage{siunitx}
\sisetup{group-digits=false}
\usepackage{cancel}

\setlength{\parindent}{0pt}
\AtBeginDocument{\setstretch{1.125}}

\title{Test 3 Review}
\author{Anthony Yu}

\begin{document}

%Some new commands
\newcommand{\problem}[1]{\subsection*{Problem {#1}}}
\newenvironment{enumAlph}{\begin{enumerate}[label=(\alph*)]}{\end{enumerate}}

\makeatletter
\newcommand{\skipitems}[1]{%
\addtocounter{\@enumctr}{#1}%
}
\makeatother

\newcommand{\chunit}[3]{\qty{#1}{{#2}\,\ce{#3}}}
\newcommand{\chuniteval}[3]{\qty[evaluate-expression]{#1}{{#2}\,\ce{#3}}}

\newtheorem{definition}{Definition}

\maketitle

\begin{definition}[Second Law of Thermodynamics]
    When energy gets converted from one to another, the amount of useful energy decreases. 
\end{definition}
An example would be that gasoline cars have only $20\%$ efficiency. 
\begin{definition}[Entropy]
    Tendency towards the loss of orderliness and the loss of useful energy.
\end{definition}

\section{ATP}
\begin{definition}[ATP]
    Nitrogen containing base adenine + sugar ribose base + 3 phospohate groups
\end{definition}
Why is breaking ATP exergonic overall? Shouldn't breaking bonds consume energy?
Remember that we're breaking it through hydrolysis. It's true that
breaking bonds take some energy, but the more stable products that from
(ADP and Pi) are much more stable, thus releasing a ton of energy. 

Kinase P transfer enzyme. Just remember it. 

\subsection{3 ways ATP does work in cell}


\section{Coupled reaction}
An example would be 

\section{Enzymes}
\subsection{How does it work}
\begin{enumerate}
    \item Induced fit model. 
    The active site is a pocket on the protein that has projecting 
    R groups (from the amino acid backbone) that forms H bonds, 
    ionic bonds, or temporary covalent bonds with the substrate. 
    The enzyme slightly changes shape as the substrate
    enters the active site, putting strain of its bonds, therefore
    reducing the activation energy.
    \item Orientation: The enzyme helps position the substrates in 
    their right 3D orientations, increasing the number of 
    productive collisions. 
\end{enumerate}

\subsection{Specificity}
The substrate must fit the active site in terms of size, shape, and 
charge compatibility. 

\subsection{5 ways cells regulate enzyme activity}
\begin{enumerate}
    \item Turn on/off genes that code enzymes.
    \item Synthesized in inactive forms.
    \item Competitive inhibition
    \item Non-competitive inhibition (also known as allosteric regulation I think?)
    \item Degradation
\end{enumerate}

\subsection{Environment}
pH affects enzymes ince the h-bonds crucial to their 3D structure
are stable in only narrow ranges of pH. 

Moderatly high temperatures optimize rxns. If the temp is too low, 
molecular motion decreases. If temp too high, h-bonds might 
be broken due to excessive motion, causing denaturation. 

\section{Peroxisomes}
They produce \ch{H2O2} as part of its natural metabolic reactions.
However it also breaks them down instantly, to prevent
harm to the cells. 

\section{Fermentation}

\subsection{CH Bonds}
Chisholm calls them "high energy bonds" not because they're hard to break.
They are actually unstable; thus when broken they have the potential
to form more stable bonds and release a lot of energy. 

\end{document}