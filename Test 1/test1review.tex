\documentclass[12pt]{article}
\usepackage{array,booktabs}
\usepackage{graphicx} % Required for inserting images
\usepackage{setspace}
\usepackage[margin=2.5cm]{geometry} % margins
\usepackage[parfill]{parskip}
\usepackage{enumitem}
\usepackage{amsfonts,latexsym,amsthm,amssymb,amsmath,amscd,euscript}

%Chemistry Packages
\usepackage{chemfig}
\usepackage{mhchem}
\usepackage{chemformula}
\usepackage{siunitx}
\sisetup{group-digits=false}
\usepackage{cancel}

\setlength{\parindent}{0pt}
\AtBeginDocument{\setstretch{1.125}}

\title{Test 1 Review}
\author{Anthony Yu}

\begin{document}

%Some new commands
\newcommand{\problem}[1]{\subsection*{Problem {#1}}}
\newenvironment{enumAlph}{\begin{enumerate}[label=(\alph*)]}{\end{enumerate}}

\makeatletter
\newcommand{\skipitems}[1]{%
\addtocounter{\@enumctr}{#1}%
}
\makeatother

\newcommand{\chunit}[3]{\qty{#1}{{#2}\,\ce{#3}}}
\newcommand{\chuniteval}[3]{\qty[evaluate-expression]{#1}{{#2}\,\ce{#3}}}

\newtheorem{definition}{Definition}

\maketitle

\part{Life}
\section{7 characteristics}
\begin{itemize}
    \item Actively maintain organized complexity
    \item Grow
    \item Reproduce
    \item Use energy and materials
    \item Evolve
    \item Sense and respond to stimuli
\end{itemize}

\section{Evolution}

\begin{definition}[Evolution]
    Change in gene frequencies in a population over time. 
\end{definition}

\part{Chemistry}
\begin{definition}[Covalent bond]
    In a covalent bond, the orbital rearranges so that electrons spend more time 
    between nuclei, causing an attraction. The nuclei get closer to each other
    until the attraction is counter-acted by their repulsion. 
\end{definition}
\begin{definition}
    An attraction between two molecules of partially opposite charges.
\end{definition}

\section{Water}
\subsection{Surface Tension}
Arises from the cohesive properties of water. The attraction between 
polar water molecules causes them to stick together and to the water beneath, 
rather than to the air, enabling it to support some weight before
it breaks. 
\subsection{Specific heat}
Water has high specific heat because the added energy first goes to 
breaking the hydrogen bonds, which takes a ton of energy,
before they increase the kinetic energy of the
molecules (which heats the water up). 
\begin{definition}[Specific heat]
    The amount of heat required to raise a gram of substance by one celcius. 
\end{definition}
\subsection{Weak bonds}
Supports capillary action and makes water a powerful solvent. 

\section{Carbon}
\begin{enumerate}
    \item forms 4 stable covalent bonds
    \item can bond to many side groups - more variety
    \item can form long chains held by stable cc bonds
    \item carbon can form single, double, triple bonds - lots of variety.
\end{enumerate}

\part{Biomolecules}
\section{Protein}
Examples: collagen, keratin, hemoglobin

\section{Lipids}
A phospholipid is ocnsisted of glycerol, two fatty acid tails, and a phosphate group
+ variable functional group. This group is hydrophilic due to the 
presence of nitrogen. 

Steroids are composed of four carbon rings. Can form hormones, cholestrol, testerone, 
etc.


\end{document}